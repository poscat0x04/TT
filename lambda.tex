\documentclass[a4paper]{book}
\usepackage{amsmath, amsthm, amssymb}
\usepackage{fancyhdr}
\usepackage{enumitem}

\title{Notes on Type Theory}
\author{Poscat}
\date{}

\pagestyle{fancy}
\lhead{Notes on Type Theory}
\rhead{\thepage}

\theoremstyle{definition}
\newtheorem{definition}{Definition}[section]
\newtheorem{lemma}{Lemma}[section]
\newtheorem{theorem}[lemma]{Theorem}
\newtheorem{notation}{Notation}[section]
\newtheorem{conv}{Convention}[section]

\newcommand{\sub}[1]{\texttt{Sub}(#1)}
\newcommand{\lterm}{$\lambda$-term }
\newcommand{\aconv}{\(\alpha\)-conversion}
\newcommand{\aeq}{=_\alpha}
\newcommand{\bred}{\(\beta\)-reduction}
\newcommand{\bconv}{\(\beta\)-conversion}
\newcommand{\bnf}{\(\beta\)-nf}
\newcommand{\beq}{=_\beta}
\newcommand{\obr}{\rightarrow_\beta}
\newcommand{\mbr}{\twoheadrightarrow_\beta}

\setenumerate[1]{label={(\arabic*)}}

\begin{document}
\maketitle
{\Large This is the notes on \textit{Type Theory and Formal Proof} taken by Poscat\\
  It is only intended to help me memorizing the concepts of type theory as no
  additional explanations will be provided}
\chapter{Untyped Lambda Calculus}
\section{Basic Definitions}
Untyped lambda calculus was invented by Church to capture the idea of ``computation" by
abstracing the operations on functions, namely application and abstraction.\\
We assume that there exist a set of all variables $V$, then we can build the set of all
lambda terms $\Lambda$ inductively from these variables.
\begin{definition}
  $\lambda$-terms, the set of all $\lambda$-terms $\Lambda$
  \begin{enumerate}
    \item (var) \text{if $u \in V$ then $u \in \Lambda$}
    \item (appl) \text{if $M \in \Lambda$ and $N \in \Lambda$ then $M N \in \Lambda$}
    \item (abst) \text{if $x \in V$ and $M \in \Lambda$ then $\lambda x. M \in \Lambda$}
  \end{enumerate}
\end{definition}
\begin{notation}
  (representation of $\lambda$-terms)\\
  We use lower case letters and their subscripted variants to represent variables,\\
  Upper case letters to represent $\lambda$-terms.\\
  \textit{Syntatically identical} $\lambda$-terms will be denoted by the triple bar symbol $\equiv$
\end{notation}
\begin{definition}
  subterms; \texttt{Sub}\\
  \begin{enumerate}
    \item \texttt{Sub}$(x) = \{x\}$, for each $x \in V$
    \item \texttt{Sub}$(M N) = \texttt{Sub}(M) \cup \texttt{Sub}(N)$ if $M \in \Lambda$ and $N \in \Lambda$
    \item \texttt{Sub}$(\lambda x.\ M)$ = \texttt{Sub}$(M)\setminus {x}$ if $M \in \Lambda$
  \end{enumerate}
\end{definition}
\begin{lemma}
  \begin{enumerate}
    \item (Reflextivity) For all $\lambda$-terms $M$, $M \in \sub{M}$
    \item (Transitivity) if $L \in \sub{M}$ and $M \in \sub{L}$ then $L \in \sub{L}$
  \end{enumerate}
\end{lemma}
\begin{definition}
  proper subterm\\
  $L$ is a proper subterm of $M$ iff $L \in \sub{M}$ and $L \not\equiv M$
\end{definition}
\begin{notation}
  \begin{itemize}
    \item[-] the outer most parenthese can be omitted
    \item[-] function applications are left associative
    \item[-] function abstractions are right associative
    \item[-] function applications have higher precedence than abstractions
  \end{itemize}
\end{notation}
\section{Free variables}
Variables in a $\lambda$-term can be categorized into 3 categories, free variables, binding variables and bound variables
\begin{definition}
  FV, free variables in a $\lambda$-term\\
  \begin{enumerate}
    \item (Variable) $FV(x) = \{x\}$,
    \item (Application) $FV(M N) = FV(M) \cup FV(N)$,
    \item (Abstraction) $FV(\lambda x.\ M) = FV(M) \setminus \{x\}$
  \end{enumerate}
\end{definition}
\begin{definition}
  Closed $\lambda$-terms\\
  A $\lambda$-term $M$ is closed iff $FV(M) = \emptyset$. A closed \lterm is also called a \textit{combinator}.
  The set of all closed \lterm s is denoted by $\Lambda^0$.
\end{definition}
\section{$\alpha$-conversion}
Renaming variables in lambda calculus is important as we don't want free variables to be accidentally captured
while substituting. We formalize this process with a relation called $\alpha$-conversion or $\alpha$-equivalence.
\begin{definition}
  (Renaming; $M^{x \rightarrow y}$, $=_\alpha$)\\
  \begin{enumerate}
    \item (Renaming) $\lambda x.\ M =_\alpha \lambda y.\ M^{x \rightarrow y}$ if $y \notin FV(M)$ and $y$ is not a binding varaible in M, where $M^{x \rightarrow y}$
          denotes every free $x$ in $M$ being replaced by $y$
    \item (Compatibility) if \(M =_\alpha N\) then \(L M =_\alpha L N\), \(M L =_\alpha N L\) and \(\lambda z.\ M =_\alpha \lambda z.\ N\)
    \item (Symmetry) if \(M =_\alpha N\) then \(N =_\alpha M\)
    \item (Reflextivity) \(M =_\alpha M\)
    \item (Transitivity) if \(L =_\alpha M\) and \(M =_\alpha N\) then \(L =_\alpha N\)
  \end{enumerate}
\end{definition}
\begin{definition}
  (\(\alpha\)-equivalent; \(\alpha\)-convertible; \(\alpha\)-variant)\\
  If \(M =_\alpha N\) then \(M\) and \(N\) are \(\alpha\)-\textit{convertible} or \(\alpha\)-\textit{equivalent}, \(M\) is a \(\alpha\)-\textit{variant} of \(N\)
\end{definition}
\begin{lemma}
  Let \(M_1 =_\alpha N_1 and M_2 =_\alpha N_2\). Then:
  \begin{enumerate}
    \item \(M_1 N_1 =_\alpha M_2 N_2\)
    \item \(\lambda x\ldotp M_1 =_\alpha \lambda x\ldotp N_1\)
    \item \(M_1[x := N_1] \equiv M_2[x := N_2]\)
  \end{enumerate}
\end{lemma}
\begin{conv}
  We identify \(\alpha\)-convertible terms, that is, \(M \equiv N\) if \(M =_\alpha N\)
\end{conv}
\section{Substitution}
In order to define \(\beta\)-reduction, we must define substitution first.
\begin{definition}
  (Subsititution)
  \begin{enumerate}
    \item[(1a)] \(x[x := N] \equiv N\)
    \item[(1b)] \(y[x := N] \equiv y\) if \(x \not\equiv y\)
    \item[(2)]  \((PQ)[x := N] \equiv (P[x := N])(Q[x := N])\)
    \item[(3)]  \((\lambda y\ldotp P)[x := N] \equiv \lambda z\ldotp (P^{y \rightarrow z}[x := N])\) if \(\lambda z\ldotp P^{y \rightarrow z}\)
          is an \(\alpha\)-variant of \(\lambda y.\ P\) such that \(z \notin FV(N)\)
  \end{enumerate}
\end{definition}
\section{\(\beta\)-reduction}
\begin{definition}
  (One step \(\beta\)-reduction, \(\rightarrow_\beta\))
  \begin{enumerate}
    \item (Basis) \((\lambda x\ldotp M) N \obr M[x := N]\)
    \item (Compatibility) if  \(M \obr N\) then \(L M \obr L N\), \(M L \obr N L\) and \(\lambda x\ldotp M \obr \lambda x\ldotp N\)
  \end{enumerate}
\end{definition}
\begin{definition}
  A term of the form \((\lambda x\ldotp M) N\) is called a \textit{redex} (\textit{red}ucible \textit{ex}pression)\\
  \(M[x := N]\) is called the \textit{contractum} of the redex
\end{definition}
\begin{definition}
  (zero-or-more-step \bred, \(\mbr\))\\
  \(M \mbr N\) if there is an \(n \geq 0\) and there are terms
  \(M_0\) to \(M_n\) such that \(M_0 \equiv M\), \(M_n \equiv N\)
  and for all \(i\) such that \(0 \leq i \leq n\) : \[
    M_i \obr M_{i+1}
    .\]
\end{definition}
\begin{lemma}
  \begin{enumerate}
    \item \(\mbr\) extends \(\obr\)
    \item \(\mbr\) is reflexive and transitive
  \end{enumerate}
\end{lemma}
\begin{definition}
  (\bconv, \(\beta\)-equality, \(\beq\))\\
  \(M \beq N\) if there is an \(n \geq  0\) and there are terms
  \(M_0\) to \(M_n\) such that \(M_0 \equiv M\), \(M_n \equiv N\)
  and for all \(i\) such that \(0 \leq i \leq n\) : \[
    \text{either }M_i \obr M_{i+1}\text{ or }M_{i+1} \obr M_i
    .\]
\end{definition}
\begin{lemma}
  \begin{enumerate}
    \item \(\beq\) extends \(\mbr\) in both directions
    \item \(\beq\) is a equivalence relation (reflexive,
          symmetric, transitive)
  \end{enumerate}
\end{lemma}
\section{Normal forms and confluence}
\begin{definition}
  (\(\beta\)-normal form; \(\beta\)-nf; \(\beta\)-normalizing)
  \begin{enumerate}
    \item \(M\) is in \(\beta\)-normal form if \(M\) does not
          contain any redex.
    \item \(M\) has \(\beta\)-normal form if there is an
          \(N\) in \(\beta\)-nf such that \(M \beq N\).
          Such an \(N\) \textit{is a \(beta\)-normal from} of \(M\)
  \end{enumerate}
\end{definition}
Define \(\Omega := (\lambda x\ldotp x\, x)(\lambda x\ldotp xx)\), then it
does not have a \(\beta\)-nf.
\begin{lemma}
  When \(M\) is in \(\beta\)-nf, then \(M \mbr N\) implies
  \(M \equiv N\)
\end{lemma}
\begin{definition}
  (Reduction path)
  \begin{enumerate}
    \item A \textit{finite reduction path} from \(M\) is a finite
          sequence of terms \(N_0, N_1, N_2, \ldots, N_n\) such that
          \(N_0 \equiv M\) and \(N_i \obr N_{i+1}\) for each
          \(0 \leq i \leq n\).
    \item A \textit{infinite reduction path} from \(M\) is an
          infinite sequence \(N_0, N_1, \ldots\) with \(N_0 \equiv M\)
          and \(N_i \obr N_{i+1}\) for each \(i \in \mathbb{N}\).
  \end{enumerate}
\end{definition}
\begin{definition}
  (Weak normalization, strong normalization)
  \begin{enumerate}
    \item \(M\) is \textit{weakly normalizing} if there is an \(N\)
          in \(\beta\)-normal form such that \(M \mbr N\)
    \item \(M\) is \textit{stronly normalizing} if there are
          no infinite reduction paths starting from \(M\)
  \end{enumerate}
\end{definition}
\begin{theorem}
  (\textit{Church-Rosser; CR; Confluence})\\
  Suppose for a given \lterm \(M\), we have \(M \mbr N_1\) and
  \(M \mbr N_2\), then there exists a \lterm \(N_3\) such that
  \(N_1 \mbr N_3\) and \(N_2 \mbr N_3\).
\end{theorem}
\begin{lemma}
  if \(M \beq N\), then there exists a \lterm  \(L_1\) such that 
  \(M \mbr L_1\) and \(N \mbr L_1\).
\end{lemma}
\begin{lemma}
  \begin{enumerate}
    \item if \(M\) has \(N\) as a \bnf, then \(M \mbr N\)
    \item A \lterm has at most one \bnf
  \end{enumerate}
\end{lemma}
\end{document}
